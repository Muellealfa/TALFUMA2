\documentclass[fleqn, 10pt]{article}

% Paquetes necesarios
\usepackage[utf8]{inputenc}
\usepackage[spanish]{babel}
\usepackage{amsthm, amsmath}
\usepackage{nccmath} %Para centrar ecuaciones
\usepackage{graphicx}
\usepackage{enumitem}

% Personalizo mi alfabeto
\DeclareMathAlphabet{\pazocal}{OMS}{zplm}{m}{n}
\newcommand{\Lb}{\pazocal{L}}

% Definimos los entornos para definiciones, teoremas, etc...
\theoremstyle{plain}
\newtheorem{proposicion}{Proposición}

\theoremstyle{definition}
\newtheorem{definition}{Definición}[section]
\newtheorem{example}{Ejemplo}[section]

%Definimos el título
\title{Teoría de Autómatas y Lenguajes Formales\\[.4\baselineskip]Práctica 1: Latex y expresiones regurales}
\author{Juan de Dios, Alfaro López}
\date{\today}

%Comienzo del documento
\begin{document}

%Generamos el título
\maketitle

\section{Relaciones}

\definition{Potencias de una relación }
\\
Dado $R \subseteq A \times A$ 
\\

$
R^n= \left\{ \begin{array}{lcc}
             R &   si  & n \leq 1 \\
             \\ \left\lbrace\left(a,b\right) : \exists x \in A ,\left(a,x\right)\in R^{n-1}\land\left(x,b\right)\in R\right\rbrace&  si & n > 1 \\
             \end{array}
   \right.
$


\example
Encuentra el conjunto de potencia $R^3$ de $R = \left\lbrace\left(1,1\right),\left(1,2\right),\left(2,3\right),\left(3,4\right)\right\rbrace$

\begin{ceqn}	%Para definir ecuación centrada en el texto
    \begin{align*} %Ecuación multilínea con alineamiento personalizado (split y align)
    R^2 = \left\lbrace\left(1,1\right),\left(1,2\right),\left(1,3\right),\left(2,4\right)\right\rbrace
    \end{align*} 
  \end{ceqn} 
  
  \begin{ceqn}	%Para definir ecuación centrada en el texto
    \begin{align*} %Ecuación multilínea con alineamiento personalizado (split y align)
    R^3 = \left\lbrace\left(1,1\right),\left(1,2\right),\left(1,3\right),\left(1,4\right)\right\rbrace
    \end{align*} 
  \end{ceqn} 






\end{document}