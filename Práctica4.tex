\documentclass[fleqn, 10pt]{article}

% Paquetes necesarios
\usepackage[utf8]{inputenc}
\usepackage[spanish]{babel}
\usepackage{amsthm, amsmath}
\usepackage{nccmath} %Para centrar ecuaciones
\usepackage{graphicx}
\usepackage{enumitem}

% Personalizo mi alfabeto
\DeclareMathAlphabet{\pazocal}{OMS}{zplm}{m}{n}
\newcommand{\Lb}{\pazocal{L}}

% Definimos los entornos para definiciones, teoremas, etc...
\theoremstyle{plain}
\newtheorem{proposicion}{Proposición}

\theoremstyle{definition}
\newtheorem{definition}{Definición}[section]
\newtheorem{example}{Ejemplo}[section]

%Definimos el título
\title{Teoría de Autómatas y Lenguajes Formales\\[.4\baselineskip]Práctica 4: TProgram Numbering and EXWHILE}
\author{Juan de Dios, Alfaro López}
\date{\today}

%Comienzo del documento
\begin{document}
\maketitle

\section{Create the simplest WHILE program that computes the diverge function (with
zero arguments) and compute the codification of its code:}

\begin{ceqn}	%Para definir ecuación centrada en el texto
    \begin{align*} %Ecuación multilínea con alineamiento
    Q: \ (0,\ 1, \ s) \ \ \ \ \ \ \ \ \\
    s: \ \ \ \ \ \ \ \ \ \ \ \  \ \ \ \ \ \ \ \ \ \ \\
    X1\ := \ X1 \ + \ 1 \ \ \  \\
    while \ X1\ != 0\  \ do \\
    X1\ := \ X1 \ + \ 1 \\
    od; \ \ \ \ \ \ \ \ \ \ \ \ \ \ \ \ \ \ \ \ \  \\
	\end{align*} 
\end{ceqn}



while2N(Q) = $\sigma^2_1$(0,code2N(c))=$\sigma^2_1$(0,$\Gamma(2,125)-1$)=$\sigma^2_1$(0,34076637)= 580608645729000\\ (un número demasiado grande que seguramente esté mal)



\end{document}