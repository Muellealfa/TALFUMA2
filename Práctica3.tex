\documentclass[fleqn, 10pt]{article}

% Paquetes necesarios
\usepackage[utf8]{inputenc}
\usepackage[spanish]{babel}
\usepackage{amsthm, amsmath}
\usepackage{nccmath} %Para centrar ecuaciones
\usepackage{graphicx}
\usepackage{enumitem}

% Personalizo mi alfabeto
\DeclareMathAlphabet{\pazocal}{OMS}{zplm}{m}{n}
\newcommand{\Lb}{\pazocal{L}}

% Definimos los entornos para definiciones, teoremas, etc...
\theoremstyle{plain}
\newtheorem{proposicion}{Proposición}

\theoremstyle{definition}
\newtheorem{definition}{Definición}[section]
\newtheorem{example}{Ejemplo}[section]

%Definimos el título
\title{Teoría de Autómatas y Lenguajes Formales\\[.4\baselineskip]Práctica 3: Turing Machine, recursive functions and WHILE language}
\author{Juan de Dios, Alfaro López}
\date{\today}

%Comienzo del documento
\begin{document}

\maketitle

\section{Define the TM solution of exercise 3.4 of the problem list and test its correct behaviour:}

\begin{ceqn}	%Para definir ecuación centrada en el texto
    \begin{align*} %Ecuación multilínea con alineamiento personalizado (split y align)
    \fbox{\includegraphics[width=250bp]{TuringMachine.jpeg}}
    \end{align*} 
  \end{ceqn}




\section{Define a recursive function for the sum of \ \ \ three values:}

 

\begin{ceqn}	%Para definir ecuación centrada en el texto
    \begin{align*} %Ecuación multilínea con alineamiento
    <\ <\pi_1^1|\sigma(\pi_3^3)>|\sigma(\pi_4^4)>
	\end{align*} 
\end{ceqn}

\section{Implement a WHILE program that computes the sum of three values. You must use an auxiliary variable that accumulates the result of the sum:} 

\begin{ceqn}	%Para definir ecuación centrada en el texto
    \begin{align*} %Ecuación multilínea con alineamiento
    Q: \ (3,\ 4, \ s) \ \ \ \ \ \ \ \ \\
    s: \ \ \ \ \ \ \ \ \ \ \ \  \ \ \ \ \ \ \ \ \ \ \\
    while \ X2\ != 0\  \ do \\
    X2\ := \ X2 \ - \ 1 \\
    X1\ := \ X1 \ + \ 1 \\
    od; \ \ \ \ \ \ \ \ \ \ \ \ \ \ \ \ \ \ \ \ \  \\
    while \ X3\ != 0\  \ do \\
    X3\ := \ X3 \ - \ 1 \\
    X1\ := \ X1 \ + \ 1 \\
    od; \ \ \ \ \ \ \ \ \ \ \ \ \ \ \ \ \ \ \ \ \  \\
	\end{align*} 
\end{ceqn}


\end{document}